A final project for the Gvahim program. This project offers a system that can store your data on multiple servers, and back them up based on the R\+A\+I\+D5 protocol (Redundant Array of Independent Disks). The system lets you manage your servers, and works perfectly even with one faulty block device.

\subsection*{Getting Started}

These instructions will get you a copy of the project up and running on your local machine for development and testing purposes.

\subsubsection*{Prerequisites}

Here are the things you will need to download in order to get this system up and running\+:


\begin{DoxyCode}
1) Download and install Python2.7
2) Download this repository on your machine (Windows coming soon)
3) Download and install any modern browser
\end{DoxyCode}


\paragraph*{Browser Issues}

Chrome\+: By default, chrome doesn\textquotesingle{}t save form data upon refresh. Therefore on \char`\"{}\+Manage Disks\char`\"{}, the refresh time might not be sufficient for user purposes. Looking into solutions.

Firefox\+: Firefox currently disables basic authentication so could be problematical and will need to add this option manually. This can be done in the about\+:config section of firefox. Looking into better solutions

\subsubsection*{Execution}

Running the servers can be done as follows\+:

W\+I\+N\+D\+O\+WS \& L\+I\+N\+UX A\+L\+I\+KE\+:

Reach parent directory (R\+A\+I\+D5) 
\begin{DoxyCode}
cd [location of RAID5]
\end{DoxyCode}
 Running Frontend Server\+: 
\begin{DoxyCode}
python -m frontend [args]
\end{DoxyCode}
 Running Block Device Server\+: 
\begin{DoxyCode}
python -m block\_device [args]
\end{DoxyCode}


Testing the servers can be done from any browser. Basic Authentication is required with common\+\_\+user and common\+\_\+password specified in frontend/config.\+ini.

\subsubsection*{Arguments}

In both the Frontend and the Block Devices, the configuration file needs to be specified as args. The Block Devices also require a bind port as shown below\+: 
\begin{DoxyCode}
python -m frontend --config-file frontend/config.ini
python -m block\_device --config-file block\_device/disks/config0.ini --bind-port 8090
python -m block\_device --config-file block\_device/disks/config1.ini --bind-port 8091
\end{DoxyCode}
 Note\+: All other Arguments are optional, see --help for help

Note\+: Default configuration files are provided in this repository. The configuration files can be created from scratch (new U\+U\+I\+Ds and such) by a python script from the parent directory (R\+A\+I\+D5)\+: 
\begin{DoxyCode}
python config\_disks.py <NUM\_OF\_BLOCK\_DEVICES>
\end{DoxyCode}
 Note\+: U\+U\+ID\textquotesingle{}s can be changed manually from the configuration files

\subsubsection*{Testing}

Another python test script from the parent directory (R\+A\+I\+D5), has also been provided, that writes to the first disk (disk0) a long sequence of numbers, for testing purposes\+: 
\begin{DoxyCode}
python test.py
\end{DoxyCode}


\subsection*{Authors}


\begin{DoxyItemize}
\item {\bfseries Roy Zohar} -\/ {\itshape Initial work} -\/ \href{https://github.com/Royz2123}{\tt My Profile}
\end{DoxyItemize}

See also the list of \href{https://github.com/Royz2123/RAID5/contributors}{\tt contributors} who participated in this project.

\subsection*{Acknowledgments}


\begin{DoxyItemize}
\item A huge thanks to Alon and Sarit for all the support 
\end{DoxyItemize}